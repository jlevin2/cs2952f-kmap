\section{Prototype}
\label{sec:prototype}
Here, we discuss our current prototype, challenges, and target applications.
Currently we compile a single shared object (\{envoy,service\}.so) for each side using compiler flags.
It is imperative we intercept syscalls on each side, knowing which side of communication the library is running on to determine which network calls to intercept.
Our library is written in C and is approximately 400 lines of code.

\subsection{\sysname Buffer}
Our implementation of \sysname buffer follows the producer/consumer model.
We map a shared structure which includes directional buffers.
We synchronize access using semaphores between the two processes.
This is critical as both applications need blocking and non-blocking, as well as signal-based calls to determine when new data has arrived.

% Used by prototype
\begin{table}[!ht]
    \begin{center}
        \resizebox{0.5\columnwidth}{!}{
            \begin{tabular}{ |c|c|c|}
                \hline
                \textbf{Function} & \textbf{Envoy} & \textbf{Service} \\
                \hline \hline
                socket & X & X \\ \hline
                connect & X &  \\ \hline
                listen &  & X \\ \hline
                accept &  & X \\ \hline
                poll &  & X \\ \hline
                select &  & X \\ \hline
                send &  & X \\ \hline
                sendto &  & X \\ \hline
                sendfile &  & X \\ \hline
                write & X & X \\ \hline
                read & X & X \\ \hline
                writev & X &  \\ \hline
                readv & X &  \\ \hline
            \end{tabular}
        }
        \caption{Library Functions Linked (HTTP/Flask only)}
        \label{t:libraries}
    \end{center}
\end{table}

\begin{table}[!ht]
    \begin{center}
        \resizebox{0.5\columnwidth}{!}{
            \begin{tabular}{ |c|c|c|}
                \hline
                \textbf{Function} & \textbf{Envoy} & \textbf{Service} \\
                \hline \hline
                socket & X & X \\ \hline
                connect & X &  \\ \hline
                listen &  & X \\ \hline
                accept &  & X \\ \hline
                write & X & X \\ \hline
                read & X & X \\ \hline
                writev & X &  \\ \hline
                readv & X &  \\ \hline
            \end{tabular}
        }
        \caption{Library Functions Linked (Tiny C Webserver only)}
        \label{t:libraries_cserver}
    \end{center}
\end{table}

\subsection{\sysname Network Calls}
The most challenging of \sysname's design is properly intercepting and replicating the behavior of network system calls.
This aspect is especially challenging because Envoy uses unique handlers for request types (UDP, TCP, HTTP\{1,2,3\}, gRPC, Quic).
Further, the service itself may use the network stack in obtuse ways.
Our investigation of Flask~\cite{flask} has revealed a number of idiosyncrasies with how Flask uses the network stack.
To make \sysname generalizable, you would truly have to intercept very single network call (or possible network related).
This task is enourmous.
For our prototype we have focused on HTTP.
In Table-\ref{t:libraries}, we show the calls on each side we link to use \sysname for flask.
Currently our prototype works with our tiny C server (Table-\ref{t:libraries_cserver}).

\subsection{Initial Results}
Blah


