\section{Discussion}
\label{sec:discussion}
Our work in Section-\ref{sec:evaluation} shows \sysname is comparable to the default system in certain circumstances and performs better in a select few.
Throughout our process, we attempted a number of optimizations which generally did not have much effect.
We will highlight them here and explain why we view that they were inconsequential.
We will also discuss future directions and learnings from the project.

\subsection{Optimizing \sysname}
Our results show the end of a number of improvements we made to \sysname.
Though no individual improvement significantly adjusted \sysname's performance.

\textbf{Compiler:} Our work began by writing the library with no compiler optimization.
Compiling with optimization flags of 02 or 03 did not adjust performance much.
The code base is relatively small and low-level so we did not expect much improvement.
Nevertheless running compile optimized code is a good practice.


\textbf{Copying:} We also experimented with the way we copied bytes.
Our naive first implementation copied at the byte level in and out of the buffer.
We improved this by using \textit{memcpy} and even experimented with multithreading \textit{memcpy}.
In either case, the improvement we saw was minimal, leading us to our cumulative conclusion that the network stack is highly optimized.

\textbf{Unsafe execution:} Another method of optimization we took was to avoid error checking in the main path of function calls.
By this, we mean that we don't null check or offset check as much as is possible.
We moved much of our initialization to the library constructor and assume it is run first (start up takes < 1 second).
The improvement this technique provided was very slight.

Overall, we feel that these optimizations were relatively surface-level.
By this, we mean that they all improve the way we communicate with our shared memory and the other processes.
They do not \textit{themselves} optimize shared memory, where we feel is causing the slow-down.

\subsection{Next Steps and Lessons}
We feel that the work here shows \sysname's promise.
Further evaluation and exploration are needed to understand under which environments \sysname performs better.
Further paths (i.e TCP/UDP/gRPC) also may show different results.
Ultimately, though, we view that this confirms how optimized the network stack is.
In principle, \sysname should be a streamlined version of the network stack, however, the decades of work built in to the kernel cannot be unappreciated.

