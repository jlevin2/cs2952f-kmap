\section{Evaluation}
\label{sec:evaluation}
\textbf{Experimental Design}:
We tested \sysname by running a single docker container with Envoy and our Tiny C Webserver~\cite{tiny}.
We then issue a series of programmatic requests to the container, targeting the Envoy port which in turn returns the content from the webserver.
We generally served fixed size randomly generated byte files, which we then compared on delivery to confirm efficacy.

\subsection{General Performance}
Our first test compares the time to transfer of 3 different size files 700B, 100KB, and 10MB.
In this experiment we use a buffer of $2^{24}$ bytes, which can support all the files.
Our results are shown in Figure-\ref{fig:results}.
We see that with smaller files \sysname is on par with the default system performance.
At higher file sizes \sysname falls behind.
Our hypothesis here is that Envoy uses readv to read data out of the buffers into iovectors.
Thus, larger files require a large number of reads, and, under the assumption our read compared to the system read is slower, this compounds \sysname's time increase.

\subsection{Buffer Size}
Our next experiment evaluated the performance of different buffer sizes.
We looked at the time to transfer for the tiny file across 2 buffer sizes $2^{16}$ and $2^{24}$ and compare those times to the system.
The results, shown in Figure-\ref{fig:buffer}, showed that the $2^{16}$ Byte buffer performed better.
We believe this is because the smaller buffer requires less pages and the memory may be more closely located since the process is not requesting as large a chunk.

